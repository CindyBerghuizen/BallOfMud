
\label{dd:save-raw}
\captionof{table}{Save the raw data of external sources}
\begin{tabular}{ l  p{10cm}}
\hline
\bf Issue & Several external sources have a limit of how far back you can request data. This means that after a given amount of time (depending on the source) you will be unable to recollect this information. As the system analyses this data and assigns a rating based on an algorithm, it may be possible that improvements or adaptations to the rating algorithm influences the ratings of incoming data and thus of previously analysed data as well. \\
\hline
\bf Decision & To be able to update the ratings when the rating calculation algorithm improves, the raw data of external sources is stored so a recalculation of the ratings can be done at a later date. \\
\hline
\bf Status & Decided \\
\hline
\bf Group & Data storage \\
\hline
\bf Assumptions & External sources limit how far back you can collect data. \\
\hline
\bf Constraints & None\\
\hline
\bf Positions & Generate a rating but do not store the raw data. \newline \newline
Generate a rating and keep the raw data for later use. \\
\hline
\bf Argument & If the algorithm that calculates the ratings is adapted and/or improved, the rating assigned to the same data item may differ. Because of this it would be impossible to recalculate the ratings if the raw data was not saved. \newline
If the data is not saved it would mean that either the ratings become a combination of ratings generated by different versions of the algorithm and thus creating an inconsistent rating, or that the previously calculated ratings should be excluded; a rating restart. \\
\hline
\bf Implications & Saving the raw data from external sources requires more storage space. \\
\hline
\bf Related decisions & None \\
\hline
\bf Related requirements & Save \\
\hline
\end{tabular}
