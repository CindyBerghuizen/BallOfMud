\section{Reflection on viewpoints}

Behind each viewpoint there have been numerous modifications which lead to the current designs. In this appendix this process is reflected upon and also the current limitations of the designs are discussed.  This add meaningful information to those reviewing the current architecture as well as to the architects themselves as it can point out flaws in the design process. The first paragraph shows the overall process for designing viewpoints while the second paragraph discusses a single view, the concurrency view. \newline

The views were thought of as an extension, a graphical overview, of some important properties within the system. First a theme was discussed that described related concerns/decisions of importance . This usually lead to numerous concrete use cases were this theme was applicable. The use cases were further refined and how they were handled was graphically displayed in the view. One advantage of this more or less bottom up approach is that it supported specific use cases instead of hand-heaving arguments. The disadvantage was that most use cases did not generally fit into one overall theme which lead to overloaded views and also that some views might be more descriptive in a more creative, non-use-case design. An example is given in the next paragraph were the concurrency view is discussed.\newline

In the concurrency view the architects tried to show the components that can work in parallel, but also the components that cannot work in parallel and form a single point of failure. The idea behind the view is to show how the performance and scalability requirements  are satisfied. So numerous use cases were identified and how these were handled was shown in the view. However as pointed in one of the review sessions, the concurrency view is overloaded. The redundancies are part of the robustness and availability of the system not concurrency. \newline 

The overall learning process regarding viewpoints is that there are multiple ways to show a theme graphically. Besides the view the rationale is also very important as it captures why this design fits the solution.