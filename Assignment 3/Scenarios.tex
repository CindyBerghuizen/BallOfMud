\section{Scenarios}

This section contains four scenarios and how the architecture responds to them. The format is obtained from \cite{clemens}. The choices for the scenarios were made on multiple factors:
\begin{enumerate}
\itemsep0em 
\item To demonstrate one of the key design decisions and the related trade-offs and sensitivity points (\ref{sec:failureb2b})
\item To indicate that some (implicit) design decisions are still open for discussion (\ref{sec:peakload},\ref{sec:privacy})
\item To test how the architecture responds to a change in an assumption and which components need to be modified (\ref{sec:bigdata})
\end{enumerate}

\subsection{Scenario: Failure in B2B/B2C} \label{sec:failureb2b}
\begin{tabularx}{\textwidth}{| l | X |}
  \hline
  \textbf{Scenario} & B2B - B2C decoupling \\
  \hline
  \textbf{Attribute} & Availability \\
  \hline
  \textbf{Environment} & Normal Operations \\
  \hline
  \textbf{Stimulus} & B2B or B2C failure \\
  \hline
  \textbf{Response} & Failure of either one does not influence the other. \\
  \hline
    &
    \begin{tabular}[t]{ | @{}| p{4cm} | l | l | l | l | @{} | }
      \hline
      \textbf{Architectural Decision} & \textbf{Sensitivity} & \textbf{Trade-off} & \textbf{Risk} & \textbf{Non-risk} \\
      \hline
      Decoupling & S1 & T1 & & \\
      \hline
      Different API & S1 & T2 & & N1 \\
      \hline
    \end{tabular}
    \\
    &  T1: When both systems are decoupled and both access the same data store, a trade-off is created. The performance of the B2C clients is directly linked to the performance of the B2B clients. \newline
    T2: The API contains duplicate functionality hence redundant code might exist. This is a trade-off between modifiability/maintainability because of the redundant code vs availability (if one system crashes, the other is not influenced). \newline
    N1: If the API was not split up, there would be a single point of failure, as mentioned by the architects. \newline
    S1: If the B2B and B2C systems are running on the same server and the server goes down, B2C and B2B will both go down. \\
  \hline
  \textbf{Reasoning} & Decoupling ensures, together with the different API, that if either the B2B or B2C module goes down, the other stays unaffected. \\
  \hline
  \textbf{Architecture Diagram} & The red API crashes. \\
   & \includegraphics[width=250px]{scenario1} \\
  \hline
\end{tabularx}

\subsection{Scenario: Peak-load on the analytics module} \label{sec:peakload}

\begin{tabularx}{\textwidth}{| l | X |}
  \hline
  \textbf{Scenario} & Peak-load on the analytics module \\
  \hline
  \textbf{Attribute} & Performance \\
  \hline
  \textbf{Environment} & Normal Operations \\
  \hline
  \textbf{Stimulus} & Recalculating ratings/reviews \\
  \hline
  \textbf{Response} & B2B latency <10 sec. \\
  \hline
    &
    \begin{tabular}[t]{ | @{}| p{4cm} | l | l | l | l | @{} | }
      \hline
      \textbf{Architectural Decision} & \textbf{Sensitivity} & \textbf{Trade-off} & \textbf{Risk} & \textbf{Non-risk} \\
      \hline
      Recalculate 1/2 times a day (Implicit) & & & R1 & \\
      \hline
      B2B accesses analytics directly (Implicit) & & T1 & & \\
      \hline
    \end{tabular}
    \\
    & R1: Recalculating the final rating once or twice a day creates a peak-load situation on the analytics module and might lead to slow or unresponsive connections between the analytics module and the B2B clients. \newline
    T1: B2B clients can directly access the analytics module, which gives the B2B clients the freedom to perform custom searches. The trade-off of this is that when the module is overloaded, the functionality may be unavailable.\\
  \hline
  \textbf{Reasoning} & Recalculating on a time frame creates a peak-load situation which may lead to an undesirable response time when a B2B user tries to perform a custom search. \\
  \hline
  \textbf{Architecture Diagram} & Analytics module overloaded. \\
   & \includegraphics[width=300px]{scenario2} \\
  \hline
\end{tabularx}

\subsection{Scenario: Review privacy} \label{sec:privacy}

\begin{tabularx}{\textwidth}{| l | X |}
  \hline
  \textbf{Scenario} & Review Privacy \\
  \hline
  \textbf{Attribute} & Privacy \\
  \hline
  \textbf{Environment} & Normal Operations \\
  \hline
  \textbf{Stimulus} & User writes a review \\
  \hline
  \textbf{Response} & Anonymous review \\
  \hline
    &
    \begin{tabular}[t]{ | @{}| p{4cm} | l | l | l | l | @{} | }
      \hline
      \textbf{Architectural Decision} & \textbf{Sensitivity} & \textbf{Trade-off} & \textbf{Risk} & \textbf{Non-risk} \\
      \hline
      Logged In & & & R1 & \\
      \hline
    \end{tabular}
    \\
    & R1: The anonymity of the user posting a review is not guaranteed. \\
  \hline
  \textbf{Reasoning} & The architecture does not state how anonymity of reviews is guaranteed, assuming that a review will be shown anonymous. \\
  \hline
  \textbf{Architecture Diagram} & Anyone can see review details. \\
   & \includegraphics[width=300px]{scenario3} \\
  \hline
\end{tabularx}

\subsection{Scenario: Big Data} \label{sec:bigdata}

\begin{tabularx}{\textwidth}{| l | X |}
  \hline
  \textbf{Scenario} & Big Data \\
  \hline
  \textbf{Attribute} & Performance, Availability \\
  \hline
  \textbf{Environment} & Normal Operations \\
  \hline
  \textbf{Stimulus} & Big Data input \\
  \hline
  \textbf{Response} & Handle all input without any loss of data. \\
  \hline
    &
    \begin{tabular}[t]{ | @{}| p{4cm} | l | l | l | l | @{} | }
      \hline
      \textbf{Architectural Decision} & \textbf{Sensitivity} & \textbf{Trade-off} & \textbf{Risk} & \textbf{Non-risk} \\
      \hline
      ETL Adapters & & & & N1 \\
      \hline
      ETL Approaches & S1 & & & \\
      \hline
      Pipeline Structure & & & R1 & \\
      \hline
    \end{tabular}
    \\
    & S1: If the amount of external input grows and becomes Big Data, the system will be unable to handle the input due to the pipe model. \newline
    N1: The adapters in the ETL are independent and therefore easily modifiable. \newline
    R1: The \emph{Filter and Store} and \emph{Extract and apply} module process all reviews sequentially which is a risk regarding scalability. \\
  \hline
  \textbf{Reasoning} & The pipeline structure won't be able to process large sets of data (Big Data) as input and won't scale easily. \\
  \hline
  \textbf{Architecture Diagram}. & \\
   & \includegraphics[width=300px]{scenario4} \\
  \hline
  \textbf{Notes} & The architectures note that \emph{a dramatically higher number of reviews would choke the system}. It was therefore an explicit choice to not handle big data, however this scenario is valuable in that it identifies the components that need to be modified in order to make it more scalable. \\
  \hline
\end{tabularx}
